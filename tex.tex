\documentclass{article}
\usepackage{amsmath}
\usepackage{amsfonts}

\begin{document}

\section*{Integral of $(3x)^2$}

Here are the step-by-step instructions to find the integral of $(3x)^2$:

\begin{enumerate}
    \item \textbf{Expand the integrand:}
    The first step is to simplify the expression inside the integral by squaring the term $(3x)$.
    $$ (3x)^2 = 3^2 \cdot x^2 = 9x^2 $$
    So, the integral becomes:
    $$ \int (3x)^2 \, dx = \int 9x^2 \, dx $$

    \item \textbf{Apply the constant multiple rule:}
    The constant multiple rule for integration states that the integral of a constant times a function is equal to the constant times the integral of the function. In this case, the constant is 9.
    $$ \int 9x^2 \, dx = 9 \int x^2 \, dx $$

    \item \textbf{Apply the power rule for integration:}
    The power rule for integration states that for any real number $n \neq -1$,
    $$ \int x^n \, dx = \frac{x^{n+1}}{n+1} + C $$
    In our case, $n = 2$. Applying the power rule, we get:
    $$ \int x^2 \, dx = \frac{x^{2+1}}{2+1} + C_1 = \frac{x^3}{3} + C_1 $$
    Note that we use $C_1$ here to distinguish it from the final constant of integration.

    \item \textbf{Combine the results:}
    Now, we multiply the result from step 3 by the constant 9 from step 2:
    $$ 9 \int x^2 \, dx = 9 \left( \frac{x^3}{3} + C_1 \right) = 9 \cdot \frac{x^3}{3} + 9 \cdot C_1 $$

    \item \textbf{Simplify the expression and the constant of integration:}
    Simplify the numerical part and combine the constant $9C_1$ into a single constant of integration, which we denote as $C$.
    $$ 9 \cdot \frac{x^3}{3} + 9 C_1 = 3x^3 + C $$

\end{enumerate}

Therefore, the integral of $(3x)^2$ is:
$$ \int (3x)^2 \, dx = 3x^3 + C $$

\end{document}